\documentclass[a4paper,9pt]{extarticle}
% General document formatting
\usepackage[margin=0.8in]{geometry}
\usepackage[parfill]{parskip}
\usepackage[utf8]{inputenc}
\usepackage{lmodern}
\usepackage{slantsc}

\usepackage{booktabs}

\newcommand{\note}[1]{\textsl{#1}}


\begin{document}
\thispagestyle{empty}

{\centering \textsc{\LARGE Anmerkungen}\\
\textsc{zum Notensatz für G. B. Biereys \textsl{\textsc{Agnus Dei}} nach L. v. Beethovens Opus 10 Nr. 1} \par}
\bigskip

Folgende Kurzübersicht listet allgemeine und spezielle Anmerkungen, die Entscheidungen zum Notensatz betreffen.
Unten folgen einige offene Punkte, die noch zu klären sind.

\section{Allgemeines}
\begin{itemize}
\item Die Reihenfolge der Instrumente wurde wie folgt geändert: Holzbläser, Horn, Streicher ohne Kontrabass, Singstimmen, Kontrabass.
\item Dynamikvorzeichnungen Kb wurde vom Cello übernommen
\item Doppelschlag vereinheitlicht: Vorzeichen immer darunter
\item Vertikale Position Dynamik vereinheitlicht: Instrumente unten, Stimmen oben
\item Fehlende N-tolen Zahlen ergänzt
\item Horizontale Position Dynamik vereinheitlicht (z.B.\ T.\ 5, 32, 35, 79, 82, 84, 102)
\end{itemize}

\section{Spezielles}

\begin{center}

\begin{tabular}{lll}
\toprule
\textbf{Takt} & \textbf{Instrument(e)} & \textbf{Anmerkung}\\

\midrule
\multicolumn{3}{c}{\textbf{Rhythmische Korrekturen}}\\
\midrule
59 & Va & 8tel statt 16tel Pause\\
63 & Fg & Punktierte Viertel \note{b} und \note{f} statt Viertel\\
91 & Fl & Untere Stimme, dritte Note \note{c} 16tel statt 8tel\\

\midrule
\multicolumn{3}{c}{\textbf{Unnötige Vorzeichnungen geklammert}}\\
\midrule
19 & Vl~1, Vc, Kb & Triolen-32tel \note{des}\\
19 & Vl~2 & 4tel \note{des}\\
28 & Kl & letzte Septolen-64tel \note{d}\\
25 & Vl~1 & 8tel \note{g}\\
39 & Va &  8tel \note{ges}\\
42 & Fg & Triolen-8tel \note{as}\\
44 & Fg & 8tel \note{des}\\
45 & Vl~2 & 8el \note{des}\\
63 & Vl~2 & 4tel \note{des}\\
67 & T & 16tel \note{as}\\
71 & Solo-S & Punktierte 16tel \note{as}\\
73 & Solo-S & Auflösungszeichen unter Doppelschlag wie in Vl~1\\
77 & Kl & 32tel \note{des}\\
82 & Vl~1 & 8tel \note{c}\\
85 & T & 8tel \note{des}\\
86 & Va & 8tel \note{ces}\\
87 & Fl \& Kl & Triolen-16tel \note{c}\\
89 & Fg & Triolen-16tel \note{des}\\

\midrule
\multicolumn{3}{c}{\textbf{Diverses}}\\
\midrule

26 & A & Vorzeichen unter Doppelschlag wie in Vl~1\\
35 & T & Fehlendes Auflösungszeichen \note{d} ergänzt\\
38 & T & Vorzeichen \note{b} zur Klarheit ergänzt\\
39 & Va & Bogen über erste Triole ergänzt wie in Vc\\
54 & Ob & Balkung zwischen \note{c} und \note{as} erggelassen\\
56 & Ob & Balkung zwischen \note{es} und \note{b} erggelassen\\
60 & Fl \& Ob & Bögen vereinheitlicht\\
60 & T & Vorzeichen \note{b} zur Klarheit ergänzt\\
66 & Fl & Balkung angeglichen wie Ob, Kl, Fg\\
66 & Fg & Vorzeichen \note{ces} ergänzt\\
95--96 & Kl & Bindebogen ergänzt wie Parallelstellen zuvor\\
101 & Fl & Dynamik-Gabeln ergänzt\\
102 & A & Vorzeichen \note{as} zur Klarheit ergänzt\\

\bottomrule
\end{tabular}

\end{center}

\end{document}
