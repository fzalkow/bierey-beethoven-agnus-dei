\documentclass[paper=A3,12pt]{scrartcl}
% General document formatting
\usepackage[margin=2in]{geometry}
\usepackage[utf8]{inputenc}
\usepackage[ngerman]{babel}
\usepackage[autostyle=true,german=quotes]{csquotes}
\usepackage{mathptmx}

\usepackage{booktabs}

\usepackage[style=authortitle-dw,library=true,nopublisher=false]{biblatex}
\renewcommand{\libraryfont}{\small}  % \small\sffamily
\bibliography{literatur}

\newcommand{\note}[1]{\textsl{#1}}


\begin{document}
\thispagestyle{empty}

\section{Einführung}

Vorliegende Edition legt erstmals die Orchesterpartitur von Gottlob Benedikt Biereys \textit{Agnus Dei} nach dem \textit{Adagio} aus Ludwig van Beethovens Klaviersonate Nr.~5 c-Moll op.~10 Nr.~1 in modernen Notenbild vor.\medskip

Anlass für die Edition war ein geplantes Konzert des Vokalensembles St. Lorenz, Nürnberg.
Der Lorenzkantor KMD Matthias Ank plante das Stück in der Orchesterfassung am 23.\ März 2020 aufzuführen, was durch die COVID-19-Pandemie 2019/20 schließlich abgesagt werden musste.
Bislang stand das Stück als moderner Notensatz nur in einer Fassung mit Orgelbegleitung durch den Carus-Verlag zur Verfügung.\footcite{2018_Schumacher_AgnusDei_Carus}
Um modernes Notenmaterial für das geplante Konzert verwenden zu können, wurde die vorliegende Edition erstellt.\medskip

Die alleinige Quelle für die Edition ist der Erstdruck von Breitkopf \& Härtel, der in digitaler Form durch die Österreichische Nationalbibliothek bereitgestellt wird.\footcite{Bierey_AgnusDei_Breitkopf}
Alle Materialien des Editionsprojektes sind in einem Github-Repositorium\footnote{\url{https://github.com/fzalkow/bierey-beethoven-agnus-dei}} unter einer Creative Commons Lizenz frei verfügbar.

\section{Zum Stück}

\begin{itemize}
\item  Hat von Euch jemand Zugriff zum New Grove? Vielleicht kann man aus dem Artikel\footcite{2001_Haertwig_Bierey_Grove} zu Bierey ein bisschen was zusammenfassen?
\end{itemize}

Die Allgemeine Musikalische Zeitung bespricht das \textit{Agnus Dei} gemeinsam mit Biereys \textit{Kyrie} nach dem ersten Satz aus Beethovens Klaviersonate Nr.~14 op.~27 Nr.~2 in cis-Moll im Juli 1832:

\begin{quote}
  So schön diese Musik bekanntlich an und für sich ist, so gut ist sie arrangirt.
  Es ist ein Agnus Dei, das hoffentlich überall Eingang haben und die Hörer erbauen wird.
  Gleich schön und dem Zwecke entsprechend ist das Kyrie.
  Der Druck dieser Partituren ist vortrefflich.
  Wir haben wohl nicht nöthig beyde Nummern erst mit beschreibenden Worten zu empfehlen:
  das Publicum wird von selbst nach ihnen verlangen und wird sich nicht täuschen.\footcite{1832_NA_Rezension_AMZ}
\end{quote}


\section{Editorische Hinweise}

Bei der vorliegenden Edition handelt es sich um eine praktische Ausgabe.
Folgende Kurzübersicht listet zunächst einige allgemeine Anmerkungen, die Entscheidungen zum Notensatz betreffen.
Anschließend werden Einzelanmerkungen zu bestimmten musikalischen Stellen gelistet.

\subsection{Allgemeines}

\begin{itemize}
  \setlength\itemsep{0.25\baselineskip}
  \item Die Reihenfolge der Instrumente wurde wie folgt geändert: Holzbläser, Horn, Streicher ohne Kontrabass, Singstimmen, Kontrabass.
  \item Die Dynamikvorzeichnungen für den Kontrabass wurden vom Cello übernommen.
  \item Die Doppelschläge vereinheitlicht, sodass das Vorzeichen immer darunter plaziert ist.
  \item Die vertikale Position der Dynamikvorzeichnung wurde vereinheitlicht: Instrumente unten, Stimmen oben.
  \item Fehlende N-tolen Zahlen wurden ergänzt.
  \item Die horizontale Position der Dynamikvorzeichnung wurde vereinheitlicht (z.B.\ T.\ 5, 32, 35, 79, 82, 84, 102).
  \item Der Text für die Gesangsstimmen wurde vom Erstdruck übernommen. Damit unterscheidet sich der Text von der Fassung mit Orgelbegleitung\footcite{2018_Schumacher_AgnusDei_Carus}, die im Carus-Verlag erhältlich ist.
\end{itemize}

\subsection{Einzelanmerkungen}

\begin{center}

\begin{tabular}{lll}
\toprule
\textbf{Takt} & \textbf{Instrument(e)} & \textbf{Anmerkung}\\

\midrule
\multicolumn{3}{c}{\textbf{Rhythmische Korrekturen}}\\
\midrule
59 & Va & 8tel statt 16tel Pause\\
63 & Fg & Punktierte Viertel \note{b} und \note{f} statt Viertel\\
91 & Fl & Untere Stimme, dritte Note \note{c} 16tel statt 8tel\\

\midrule
\multicolumn{3}{c}{\textbf{Unnötige Vorzeichnungen geklammert}}\\
\midrule
19 & Vl~1, Vc, Kb & Triolen-32tel \note{des}\\
19 & Vl~2 & 4tel \note{des}\\
28 & Kl & letzte Septolen-64tel \note{d}\\
25 & Vl~1 & 8tel \note{g}\\
39 & Va &  8tel \note{ges}\\
42 & Fg & Triolen-8tel \note{as}\\
44 & Fg & 8tel \note{des}\\
45 & Vl~2 & 8el \note{des}\\
63 & Vl~2 & 4tel \note{des}\\
67 & T & 16tel \note{as}\\
71 & Solo-S & Punktierte 16tel \note{as}\\
73 & Solo-S & Auflösungszeichen unter Doppelschlag wie in Vl~1\\
77 & Kl & 32tel \note{des}\\
82 & Vl~1 & 8tel \note{c}\\
85 & T & 8tel \note{des}\\
86 & Va & 8tel \note{ces}\\
87 & Fl \& Kl & Triolen-16tel \note{c}\\
89 & Fg & Triolen-16tel \note{des}\\

\midrule
\multicolumn{3}{c}{\textbf{Diverses}}\\
\midrule

26 & A & Vorzeichen unter Doppelschlag wie in Vl~1\\
35 & T & Fehlendes Auflösungszeichen \note{d} ergänzt\\
38 & T & Vorzeichen \note{b} zur Klarheit ergänzt\\
39 & Va & Bogen über erste Triole ergänzt wie in Vc\\
54 & Ob & Balkung zwischen \note{c} und \note{as} erggelassen\\
56 & Ob & Balkung zwischen \note{es} und \note{b} erggelassen\\
60 & Fl \& Ob & Bögen vereinheitlicht\\
60 & T & Vorzeichen \note{b} zur Klarheit ergänzt\\
66 & Fl & Balkung angeglichen wie Ob, Kl, Fg\\
66 & Fg & Vorzeichen \note{ces} ergänzt\\
95--96 & Kl & Bindebogen ergänzt wie Parallelstellen zuvor\\
101 & Fl & Dynamik-Gabeln ergänzt\\
102 & A & Vorzeichen \note{as} zur Klarheit ergänzt\\

\bottomrule
\end{tabular}

\setlength\bibitemsep{0.25\baselineskip}  % {1.5\itemsep}
\printbibliography[heading=bibnumbered]

\end{center}

\end{document}
